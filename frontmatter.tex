
\raggedbottom
\thispagestyle{empty}
\maketitle{}

\begin{abstract}
\begin{center}\textbf{Summary}\end{center}
The Design \& Make exercise introduces Engineering Design Process Models, Principles of Design \& Manufacture, and Project Management techniques. 
The lectures provides a overview of the techniques that are typically employed whilst the exercise provides an opportunity to put some of these techniques into practice as you design a product to meet a design brief. 
You will then manufacture and assemble your design in the 2nd year. This document contains the course notes that support the exercise and lectures.
\end{abstract}

\frontmatter
\cleardoublepage{}
\setcounter{tocdepth}{2}
\setcounter{page}{1}
\tableofcontents

\cleardoublepage{}
\listoffigures

\cleardoublepage{}
\listoftables

\cleardoublepage{}
\section*{List of Acronyms}
\begin{acronym}[TDMA]
  \acro{BOM}{Bill of Materials}
  \acro{CAD}{Computer Aided Design}
  \acro{CNC}{Computer Numerically Controlled}
  \acro{CPM}{Critical Path Methods}
  \acro{DfA}{Design for Assembly}
  \acro{DfM}{Design for Manufacture}
  \acro{DfX}{Design for X}
  \acro{DMLS}{Direct Metal Laser Sintering}
  \acro{FDM}{Fused Deposition Modelling}
  \acro{FFF}{Fused Filament Fabrication}
  \acro{MCDA}{Multi-Criteria Decision Analysis}
  \acro{NASA}{National Aeronautics and Space Administration}
  \acro{PDS}{Product Design Specification}
  \acro{PERT}{Project Evaluation and Review Techniques}
  \acro{PLM}{Product Lifecycle Management}
  \acro{SLM}{Selective Laser Melting}
  \acro{SLS}{Selective Laser Sintering}
  \acro{STL}{Stereolithography}
  \acro{VDI2221}{Systematic Design Process Model}
\end{acronym}

\cleardoublepage{}
\mainmatter